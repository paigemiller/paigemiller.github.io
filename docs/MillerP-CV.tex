% LaTeX Curriculum Vitae Template
%
% Copyright (C) 2004-2009 Jason Blevins <jrblevin@sdf.lonestar.org>
% http://jblevins.org/projects/cv-template/
%
% You may use use this document as a template to create your own CV
% and you may redistribute the source code freely. No attribution is
% required in any resulting documents. I do ask that you please leave
% this notice and the above URL in the source code if you choose to
% redistribute this file.

\documentclass[letterpaper]{article}

\usepackage{hyperref}
\usepackage{geometry}
\usepackage{marvosym} % for symbols like \LARGE{\PointingHand} \normalsize 
\usepackage{multicol} 
\usepackage{marvosym}

% Comment the following lines to use the default Computer Modern font
% instead of the Palatino font provided by the mathpazo package.
% Remove the 'osf' bit if you don't like the old style figures.
\usepackage[T1]{fontenc}
\usepackage[sc,osf]{mathpazo}

% Set your name here
\def\name{Paige B. Miller}

% Replace this with a link to your CV if you like, or set it empty
% (as in \def\footerlink{}) to remove the link in the footer:
\def\footerlink{}

% The following metadata will show up in the PDF properties
\hypersetup{
  colorlinks = true,
  urlcolor = black,
  pdfauthor = {\name},
  pdfkeywords = {ecology, statistics, epidemiology},
  pdftitle = {\name: Curriculum Vitae},
  pdfsubject = {Curriculum Vitae},
  pdfpagemode = UseNone
}

\geometry{
  body={6.5in, 8.5in},
  left=1.0in,
  top=1.25in
}

% Customize page headers
\pagestyle{myheadings}
\markright{\name}
\thispagestyle{empty}

% Custom section fonts
\usepackage{sectsty}
\sectionfont{\rmfamily\mdseries\Large}
\subsectionfont{\rmfamily\mdseries\itshape\large}

% Other possible font commands include:
% \ttfamily for teletype,
% \sffamily for sans serif,
% \bfseries for bold,
% \scshape for small caps,
% \normalsize, \large, \Large, \LARGE sizes.

% Don't indent paragraphs.
\setlength\parindent{0em}

% Make lists without bullets
\renewenvironment{itemize}{
  \begin{list}{}{
    \setlength{\leftmargin}{1.5em}
  }
}{
  \end{list}
}

\begin{document}

% Place name at left
{\huge \name}

% Alternatively, print name centered and bold:
%\centerline{\huge \bf \name}

\vspace{0.25in}

\begin{minipage}{0.45\linewidth}
  \href{}{Research Assistant} \\
  Odum School of Ecology \\
  University of Georgia \\
  Athens, GA 30605
\end{minipage}
\begin{minipage}{0.45\linewidth}
  \begin{tabular}{ll}
    Phone: & (651) 767-2412 \\
    Email: & \href{mailto:paige.miller@uga.edu}{\tt paigemiller554@gmail.com} \\
    Homepage: & \href{https://paigemiller.github.io}{\tt https://paigemiller.github.io} \\
  \end{tabular}
\end{minipage}

\section*{Education}

\begin{itemize}
  \item \textbf{Ph.D} University of Georgia (Athens, GA) 	\hfill \textit{2015 - 2020}
  \item \textbf{B.A.} Math and Biology, Gustavus Adolphus College (Saint Peter, MN) \hfill  \textit{2011 - 2015}
\end{itemize}

\section*{Experience}

\begin{itemize}
  \item  \textbf{Intern} Division of Global Migration and Quarantine, CDC (Atlanta, Georgia)
   \hfill  \textit{2017}
   \\ Data visualization and analysis of Electronic Data Notification system with the broader goal of improving data quality control and investigating trends in notifiable conditions.
  \item  \textbf{Intern} Division of HIV/AIDS Prevention, CDC (Atlanta, Georgia)   \hfill  \textit{2015}
  \\ Data analysis for a project examining the efficacy of a brief messaging program about HIV testing among men who have sex with men. 
\end{itemize}

\section*{Presentations and Publications}

\subsection*{Journal Articles}

\begin{itemize}
\item    \textbf{Miller PB} and JM Drake. The effects of core-periphery network structure on disease spread. \textit{In Prep for BMC Theoretical Biology and Medical Modeling}. 
\item    \textbf{Miller PB}, Kennedy C, Vasquez D, Bui T, King J, Lewis V, Reynolds WC, Thomas O,  Wenclawiak J, Drake JM. Characteristics of COVID-19 Outbreaks in Care, Correctional, and Food Processing Facilities in the United States. \textit{In Prep for Emerging Infectious Diseases}. 
\item  Juliana Taube, \textbf{Miller PB}, JM Drake. An open-access database of transmission trees used to explore superspreader epidemiology. \textit{In Review at Plos Biology}. 
\item    \textbf{Miller PB},  Whalen CC, and JM Drake. Biology or Behavior? Effects of sex-traits and assortative mixing on male-bias in Tuberculosis. \textit{In Review at Royal Society Open Science}. 
\item   \textbf{Miller PB}, Zalwango S, Galiwango R, Kakaire R, Sekandi J, Steinbaum L, Drake JM, Whalen CC, and N Kiwanuka.Tuberculosis spread in social networks: A cross-sectional study of TB in Kampala, Uganda. \textit{In review at BMC Infectious Diseases}. 
\item Drake JM, Brett TS,  Chen S, Epureanu B, Ferrari M, Marty E, \textbf{Miller PB}, O'Dea EB, O'Regan SM, Park AW, and P Rohani. The statistics of epidemic transitions. 2018. \textit{PLoS Computational Biology}. 
\item Brett TS,  O'Dea EB, Marty E, \textbf{Miller PB}, Park AW, Drake JM, and P Rohani. Anticipating epidemic transitions with imperfect data. 2018. \textit{PLoS Computational Biology}. 
\item \textbf{Miller PB}, O'Dea EB, Rohani P, and JM Drake. Forecasting infectious disease emergence subject to seasonal forcing. 2017. \textit{BMC Theoretical Biology and Medical Modeling}.
\item Bloch Qazi M, \textbf{Miller PB}, Poeschel P, Phan MH, Thayer JL, Medrano CL, and MC Bloch Qazi. Transgenerational effects of maternal and grandmaternal age on offspring viability and performance in Drosophila melanogaster. 2017. \textit{Journal of Insect Physiology}.
\item Mansergh G, \textbf{Miller PB}, JH Herbst, MJ Mimiaga, and J Holman. Effects of Brief Messaging about Undiagnosed Infections Detected through HIV Testing among Black and Latino Men who have Sex with Men in the United States. 2015. \textit{Sexually Transmitted Diseases}.
\item \textbf{Miller PB}, Obrik-Uloho OT, Phan MH, Medrano CL, Renier JS, Thayer JL, Wiessner G, and MC Bloch Qazi. 2014. The Song of the Old Mother: Reproductive Senescence in Female Drosophila. \textit{FLY}.
\end{itemize}

\subsection*{Presentations}
\begin{itemize}
\item  \textbf{Miller PB}, Whalen CC, Noah Kiwanuka, and JM Drake. 2019. Tuberculosis transmission and social network structure: A case study in Kampala, Uganda and simulations on structured networks. Epidemics. Charleston, SC. 
\item  \textbf{Miller PB}, Whalen CC, and JM Drake. 2019. Can social network patterns explain male-bias in TB cases? Center for the Ecology of Infectious Diseases Annual Meeting. Athens, GA.
\item \textbf{Miller PB}. 2018. Data visualization in R and producing professional quality graphics. R Ladies Meeting. Athens GA. 
\item   \textbf{Miller PB}, Houck K, and JM Drake. 2018. Age-targeted interventions for TB in endemic regions. National Science Foundation Research Training Program Meeting. Arlington, VA. 
\item  \textbf{Miller PB} and JM Drake. 2017. Spatial pattern formation of an infectious disease on the verge of elimination. Odum School of Ecology Graduate Student Research Symposium, Athens, GA. 
\item \textbf{Miller PB} and JM Drake. 2016. Forecasting infectious diseases with early warning signals. MIDAS Symposium, Reston, VA. 
\item \textbf{Miller PB}. and JM Drake. 2015. Using the power ratio as an early warning statistic for predicting emerging infectious disease outbreaks. National Science Foundation, Emerging Researchers National Conference, Washington D.C.
\item \textbf{Miller PB}. and G Mansergh. 2015. Effects of Brief Messaging about Undiagnosed Infections Detected through HIV Testing among Black and Latino Men who have Sex with Men in the United States. Celebration of Creative Inquiry, Gustavus Adolphus College.
\end{itemize}

\section*{Grants \& Awards}
\begin{itemize}
\item Infectious Disease Ecology Across Scales Grant (\$1,000) \hfill \textit{2018} 
\item ARCS Foundation, Graduate Research Scholarship (Honorable mention) \hfill \textit{2017} 
\item National Science Foundation Graduate Research Fellowship (\$102,000) \hfill \textit{2016} 
\item Gustavus Adolphus College Award for Outstanding Student in Public Health \hfill \textit{2015} 
\end{itemize}

\section*{Graduate coursework}
\begin{multicols}{2}
\begin{itemize}
\item Population Ecology
\item Probability and Statistics I \& II
\item Applied Linear Regression
\item Time Series Analysis 
\item Machine Modeling
\item Multi-scale modeling
\item Bayesian Statistics
\item Fundamentals of Disease Ecology I \& II
\item IDEAS Collaborative Research Capstone 
\end{itemize}
\end{multicols}

\section*{Service}
\begin{itemize}
\item \textbf{Journal refereeing}: American Journal of Epidemiology, Current Bioinformatics, Proceedings of the National Academy of Sciences (co-reviewer), Nature Scientific Data (co-reviewer)
\item \textbf{Mentoring}: Sunishka Thakur (Women in Science Program, 2015-2016), Kennedy Houck (Population Biology of Infectious Diseases REU Program, 2018), Julianna Taube (Population Biology of Infectious Diseases REU Program, 2019), Culzean Kennedy (Student Research Program at UGA, 2020) and Cody Reynolds, Vanessa Lewis, Jessica Wenclawiak, Olivia Thomas, Tommy Bui (Coronavirus working group, 2020)
\item \textbf{Outreach}: HIV testing volunteer (LiveForward, 2015-current),  Treasurer (Odum School of Ecology Graduate Student Organization, 2016-2017), Head coach and board member (Classic City Volleyball Club, 2017-current.) 
\end{itemize}

\section*{Affiliations}
\begin{itemize}
\item Coronavirus Working Group, University of Georgia
\item Infectious Disease Ecology Across Scales Program, University of Georgia  
\item Center for the Ecology of Infectious Diseases, University of Georgia 
\item Epidemiology in Action Research Group, University of Georgia 
\item R Ladies Group, Athens GA 
\item Modeling Infectious Disease Agent Study (MIDAS) Network

\end{itemize}

\bigskip

% Footer
\begin{center}
  \begin{footnotesize}
    Last updated: \today \\
    \href{\footerlink}{\texttt{\footerlink}}
  \end{footnotesize}
\end{center}

\end{document}
